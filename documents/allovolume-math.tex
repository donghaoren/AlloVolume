\documentclass[letter, 11pt]{article}

\newcommand{\mtitle}{The Mathematics of AlloVolume}
\newcommand{\mauthor}{Donghao Ren}

\usepackage[top=1.5in, bottom=1.5in, left=1in, right=1in]{geometry}
\usepackage{amsmath}
\usepackage{fancyhdr}
\usepackage{hyperref}
\usepackage{graphicx}
\setlength{\headheight}{15.2pt}
\pagestyle{fancy}
\fancyhead{}
\fancyfoot{}
\fancyhead[L]{\mtitle}
\fancyhead[R]{\mauthor}
\fancyfoot[C]{\thepage}

% \usepackage{fourier} % math & rm
% \usepackage[scaled=0.875]{helvet} % ss
% \renewcommand{\ttdefault}{lmtt} %tt

\begin{document}

\thispagestyle{fancy}

\centerline{\Large\textbf{\mtitle}}
\vspace{0.1in}
\centerline{\large{\mauthor}}

\setlength{\parindent}{0cm}

\section{Differential Blending}

Here we derive the formulas for differential blending in AlloVolume.

\subsection{Alpha Blending Equations}

Here is the alpha blending equation taken from Wikipedia:
\begin{equation*}
\begin{cases}
\mathrm{out}_{\mathrm{A}} = \mathrm{src}_{\mathrm{A}} + \mathrm{dst}_{\mathrm{A}} (1 - \mathrm{src}_{\mathrm{A}}) \\
\mathrm{out}_{\mathrm{RGB}} = \bigl( \mathrm{src}_{\mathrm{RGB}} \mathrm{src}_{\mathrm{A}} + \mathrm{dst}_{\mathrm{RGB}} \mathrm{dst}_{\mathrm{A}} \left( 1 - \mathrm{src}_{\mathrm{A}} \right) \bigr) \div \mathrm{out}_{\mathrm{A}} \\
\mathrm{out}_{\mathrm{A}} = 0 \Rightarrow \mathrm{out}_{\mathrm{RGB}} = 0
\end{cases}
\end{equation*}

With premultiplied alpha $ \mathrm{out}_{\mathrm{PMRGB}} = \mathrm{out}_{\mathrm{RGB}} \times \mathrm{out}_{\mathrm{A}} $, we have:
\begin{equation*}
\begin{cases}
\mathrm{out}_{\mathrm{A}} = \mathrm{src}_{\mathrm{A}} + \mathrm{dst}_{\mathrm{A}} (1 - \mathrm{src}_{\mathrm{A}}) \\
\mathrm{out}_{\mathrm{PMRGB}} = \mathrm{src}_{\mathrm{PMRGB}} + \mathrm{dst}_{\mathrm{PMRGB}} \left( 1 - \mathrm{src}_{\mathrm{A}} \right)
\end{cases}
\end{equation*}

The color of each pixel in volume rendering is given by a sequence of blending along a ray $\vec{p}(t) = \vec{s} + t \vec{d}$.
Take the voxel values $V(\vec{p}(t))$ at the time steps $t$ from the end of the ray to the start,
apply the transfer function $T(V(\vec{p}(t)))$ to get the RGBA values, and then blend them together.
To do this, we can adjust the alpha for the transfer function according to the step size,
or use the differential version as described below.

From the blending formula with premultiplied alpha, we can derive the differential version:
\begin{align*}
\alpha(t, \alpha_0) &= 1 - (1 - A) ^ {\tfrac{t}{L}} + (1 - A) ^ {\tfrac{t}{L}} \alpha_0 \\
c(t, c_0) &= \bigl(1 - (1 - A) ^ {\tfrac{t}{L}}\bigr) \frac{\mathrm{PMRGB}}{\mathrm{A}} + (1 - A) ^ {\tfrac{t}{L}} c_0
\end{align*}

\newcommand{\A}{\mathrm{A}}
\newcommand{\RGB}{\mathrm{RGB}}
\newcommand{\dd}{\mathrm{d}}

Take the derive at $t = 0$, we have the differential equations:
\begin{align}
\alpha'(t) &= \frac{\bigl(\alpha(t)-1\bigr)\ln\bigl(1-\A(t)\bigr)}{L} \\
c'(t) &= \frac{\bigl(c(t)-\RGB(t)\bigr)\ln\bigl(1-\A(t)\bigr)}{L}
\end{align}

\subsection{The Pre-Integration Technique}

Pre-integration can help reduce artifacts in the rendering results.

Let's assume we are wokring on a segment of the ray, we have volume values $v_0$ and $v_1$,
transfer functions $T_\RGB$, $T_\alpha$, initial color $c_0$ and alpha $\alpha_0$, blending length $s$.

The differential function is now:
\begin{align*}
\alpha'(t) &= \frac{\bigl(\alpha(t)-1\bigr)\ln\bigl(1-T_\alpha((1-\frac{t}{s})v_0+\frac{t}{s}v_1)\bigr)}{L} \\
c'(t) &= \frac{\bigl(c(t)-T_\RGB((1-\frac{t}{s})v_0+\frac{t}{s}v_1)\bigr)\ln\bigl(1-T_\alpha((1-\frac{t}{s})v_0+\frac{t}{s}v_1)\bigr)}{L} \\
\alpha(0) &= \alpha_0 \,,\; c(0) = c_0
\end{align*}

We want to find $\alpha(s)$ and $c(s)$. Now we substitute $t$ by $ts$:
\begin{align*}
\alpha_1(t) &= \alpha(ts) \,,\; c_1(t) = c(ts)\\
\alpha_1'(t) &= s\alpha'(ts) \,,\; c_1'(t) = s c'(ts)\\
\alpha_1'(t) &= s\frac{\bigl(\alpha_1(t)-1\bigr)\ln\bigl(1-T_\alpha((1-t)v_0+tv_1)\bigr)}{L} \\
c_1'(t) &= s\frac{\bigl(c_1(t)-T_\RGB((1-t)v_0+tv_1)\bigr)\ln\bigl(1-T_\alpha((1-t)v_0+tv_1)\bigr)}{L} \\
\alpha_1(0) &= \alpha_0 \,,\; c_1(0) = c_0
\end{align*}

From now on, $\alpha, c$ mean $\alpha_1, c_1$, so we want to get $\alpha(1)$ and $c(1)$.

To make things shorter, we define:
\begin{align*}
T_\alpha(v_0, v_1, t) &= T_\alpha\bigl((1-t)v_0+t v_1\bigr) \\
T_\RGB(v_0, v_1, t) &= T_\RGB\bigl((1-t)v_0+t v_1\bigr)
\end{align*}

Rewrite the equations:
\begin{align*}
\alpha'(t) - \frac{s}{L} \ln\bigl(1-T_\alpha(v_0, v_1, t)\bigr) \alpha(t) &= -\frac{s}{L} \ln\bigl(1-T_\alpha(v_0, v_1, t)\bigr) \\
c'(t) - \frac{s}{L} \ln\bigl(1-T_\alpha(v_0, v_1, t)\bigr) c(t) &= -\frac{s}{L} T_\RGB(v_0, v_1, t) \ln\bigl(1-T_\alpha(v_0, v_1, t)\bigr)
\end{align*}

The differential blending equations is a first-order, linear, inhomogeneous, ODE with function coefficients, which can be written as:
\begin{equation*}
\frac{\dd y}{\dd x} + P(x)y = Q(x)
\end{equation*}

The solution is given by:
\begin{align*}
\alpha(1) &= e^{\frac{s}{L}\int_0^1  \ln\bigl(1-T_\alpha(v_0, v_1, \lambda)\bigr) \, \dd\lambda}\left[-\frac{s}{L}\int_0^1 e^{-\int_0^\lambda \frac{s}{L} \ln\bigl(1-T_\alpha(v_0, v_1, \epsilon)\bigr) \, \dd\epsilon} \ln\bigl(1-T_\alpha(v_0, v_1, \lambda)\bigr) \, {\dd\lambda} +\alpha_0 \right] \\
c(1) &= e^{\frac{s}{L}\int_0^1  \ln\bigl(1-T_\alpha(v_0, v_1, \lambda)\bigr) \, \dd\lambda}\left[-\frac{s}{L}\int_0^1 e^{-\int_0^\lambda \frac{s}{L} \ln\bigl(1-T_\alpha(v_0, v_1, \epsilon)\bigr) \, \dd\epsilon} T_\RGB(v_0, v_1, t) \ln\bigl(1-T_\alpha(v_0, v_1, \lambda)\bigr) \, {\dd\lambda} + c_0 \right]
\end{align*}

For simplicity, we define:
\begin{align}
M_\alpha(v_0, v_1, s) &= e^{\frac{s}{L}\int_0^1  \ln\bigl(1-T_\alpha(v_0, v_1, \lambda)\bigr) \, \dd\lambda} \\
M_\RGB(v_0, v_1, s) &= -\frac{s}{L}\int_0^1 e^{-\int_0^\lambda \frac{s}{L} \ln\bigl(1-T_\alpha(v_0, v_1, \epsilon)\bigr) \, \dd\epsilon} T_\RGB(v_0, v_1, t) \ln\bigl(1-T_\alpha(v_0, v_1, \lambda)\bigr) \, {\dd\lambda}
\end{align}

The solution becomes:
\begin{align}
\alpha(1) &= M_\alpha(v_0, v_1, s) \alpha_0 + \bigl(1 - M_\alpha(v_0, v_1, s)\bigr) \alpha_0 \label{eq:malpha2eliminate} \\
c(1) &= M_\alpha(v_0, v_1, s) \big(M_\RGB(v_0, v_1, s) + c_0\big)
\end{align}

\autoref{eq:malpha2eliminate} holds because $\alpha_0 = 1 \Rightarrow \alpha(1) = 1$. therefore the second integral can be removed.

To pre-integrate, we only need to compute $M_\alpha(v_0, v_1, s)$ and $M_\RGB(v_0, v_1, s)$ from the transfer function.
$M_\alpha(v_0, v_1, s)$ can be written as the following:
\begin{align}
M_\alpha(v_0, v_1, s) &= e^{\frac{s}{L}\frac{1}{v_1-v_0}\int_{v_0}^{v_1} \ln\bigl(1-T_\alpha(t)\bigr) \, \dd t} \\
M_\alpha(v_0, v_1) &= e^{\frac{1}{v_1-v_0}\int_{v_0}^{v_1} \ln\bigl(1-T_\alpha(t)\bigr) \, \dd t} \\
M_\alpha(v_0, v_1, s) &= M_\alpha(v_0, v_1)^{\tfrac{s}{L}}
\end{align}

Therefore we only have to compute $M_\alpha(v_0, v_1)$, which is independent of $s$ and $L$.
$M_\RGB(v_0, v_1, s)$ can be written as the following:
\begin{align}
\mathrm{Let} & \,\, Y(v) = \int_0^t\ln(1 - T_\alpha(v)) \, \dd t \\
\mathrm{Then} & \,\, \int_0^\lambda \ln\bigl(1-T_\alpha(v_0, v_1, \epsilon)\bigr) \, \dd\epsilon = \frac{Y(v_0 + \lambda (v_1 - v_0)) - Y(v_0)}{v_1 - v_0} \\
M_\RGB(v_0, v_1, s) &= -\frac{s}{L}\frac{1}{v_1-v_0}\int_{v_0}^{v_1} e^{-\frac{s}{L}\frac{Y(t)-Y(v_0)}{v_1 - v_0}} T_\RGB(t) \ln\bigl(1-T_\alpha(t)\bigr) \, {\dd t} \\
  &= -\frac{s}{L}\frac{1}{v_1-v_0}e^{\frac{s}{L}\frac{Y(v_0)}{v_1 - v_0}}\int_{v_0}^{v_1} e^{-\frac{s}{L}\frac{Y(t)}{v_1 - v_0}} T_\RGB(t) \ln\bigl(1-T_\alpha(t)\bigr) \, {\dd t} \\
\end{align}

To furture reduce it, let:
\begin{equation*}
p = \frac{s}{L}\frac{1}{v_1 - v_0}
\end{equation*}

We then have:
\begin{equation*}
M_\RGB(v_0, v_1, s) = -p e^{p Y(v_0)} \int_{v_0}^{v_1}{ e^{-p Y(t)} }T_\RGB(t) \ln\bigl(1-T_\alpha(t)\bigr) \, {\dd t}
\end{equation*}

Let:
\begin{equation*}
Z(v, p) = \int_{0}^{v}{ e^{-p Y(t)} }T_\RGB(t) \ln\bigl(1-T_\alpha(t)\bigr) \, {\dd t}
\end{equation*}

We have:
\begin{equation*}
M_\RGB(v_0, v_1, s) = -p e^{p Y(v_0)} \big(Z(v_1, p) - Z(v_0, p)\big)
\end{equation*}

Now the only thing we need in order to compute $M_\RGB(v_0, v_1, s)$ is $Y(v)$ and $Z(v, p)$.

Also, $M_\alpha(v_0, v_1, s)$ is given by:
\begin{equation*}
M_\alpha(v_0, v_1, s) = e^{p \big(Y(v_1)-Y(v_0)\big)}
\end{equation*}

Putting it altogether, first we compute $Y(v)$ and $Z(v, p)$:
\begin{align*}
Y(v) &= \int_0^v\ln(1 - T_\alpha(t)) \, \dd t \\
Z(v, p) &= \int_{0}^{v}{ e^{-p Y(t)} }T_\RGB(t) \ln\bigl(1-T_\alpha(t)\bigr) \, {\dd t}
\end{align*}

Then we have:
\begin{align*}
M_\alpha(v_0, v_1, s) &= e^{p \big(Y(v_1)-Y(v_0)\big)} \\
M_\RGB(v_0, v_1, s) &= -p e^{p Y(v_0)} \big(Z(v_1, p) - Z(v_0, p)\big)
\end{align*}

The blending equation is:
\begin{align*}
\alpha_{v_1} &= M_\alpha(v_0, v_1, s) \alpha_{v_0} + \bigl(1 - M_\alpha(v_0, v_1, s)\bigr) \alpha_{v_0} \label{eq:malpha2eliminate} \\
c_{v_1} &= M_\alpha(v_0, v_1, s) \big(M_\RGB(v_0, v_1, s) + c_{v_0}\big)
\end{align*}



\end{document}

